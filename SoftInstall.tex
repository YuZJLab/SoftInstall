\documentclass[]{article}
\usepackage[backref,bookmarks=true,bookmarksopen=true,pdftitle=SoftInstall,pdfauthor=YU Zhejian]{hyperref}
\usepackage{times}
\usepackage{indentfirst}
\renewcommand\familydefault{\sfdefault}
\usepackage[inner=0.5in,outer=0.5in]{geometry}
\usepackage{titletoc}
\titlecontents{part}[2em]{\addvspace{4pt}\large}{\contentslabel[\thecontentslabel]{4em}}{}{\ \titlerule*[.5pc]{.}\ \thecontentspage}[]
\titlecontents{section}[1em]{\addvspace{2pt}}{\bfseries\contentslabel[\thecontentslabel]{4em}}{}{\ \titlerule*[.5pc]{.}\ \thecontentspage}[]

\begin{document}
\title{\fontsize{40}{40}\selectfont\scshape Fantastic Software\\{\Huge and}\\Where to Find Them}
\author{YU Zhejian}
\maketitle
This book involves the software used in the previous sections, the commands they provided and the way to install them.
\tableofcontents
\newpage

\part{Read Data Pre-Processing}
\section{SRA-Toolkit}

Conda: \url{https://anaconda.org/bioconda/sra-tools}

Official: \url{https://trace.ncbi.nlm.nih.gov/Traces/sra/sra.cgi?view=software}

GitHub: \url{https://github.com/ncbi/sra-tools/}

License: PUBLIC DOMAIN NOTICE

C 70.4\% C++ 19.7\% Makefile 4.0\% Shell 2.0\% Perl 1.8\ Python 1.0\% Other 1.1\%



\section{htslib}

GitHub: \url{https://github.com/samtools/htslib} with MIT/Expat license, Modified 3-Clause BSD license

Conda: \url{https://anaconda.org/bioconda/htslib} with MIT License

Official: \url{http://www.htslib.org/}

C 89.2\% C++ 3.8\% Perl 2.7\% Makefile 1.4\% Roff 1.2\% M4 1.0\% Other 0.7\%



\section{Tabix}

Conda: \url{https://anaconda.org/bioconda/tabix}

GitHub: \url{https://github.com/samtools/tabix}

SourceForge: \url{https://sourceforge.net/projects/samtools}

License: BSD

C 83.8\% Java 6.8\% Python 4.6\% TeX 2.8\% Perl 2.0\%


\part{Read Quality Controller}

\section{QualiMap}

Official: \url{http://qualimap.bioinfo.cipf.es/}

BitBucket: \url{https://bitbucket.org/kokonech/qualimap/}

License: GPL 2

Java

García-Alcalde, F., Okonechnikov, K., Carbonell, J., Cruz, L. M., Götz, S., Tarazona, S., Dopazo, J., Meyer, T. F. and Conesa, A. (2012) Qualimap: evaluating next-generation sequencing alignment data, Bioinformatics, 28(20), pp. 2678–2679. doi: 10.1093/bioinformatics/bts503.

\begin{verbatim}
@article{Qualimap1,
abstract = {Motivation: The sequence alignment/map (SAM) and the binary
alignment/map (BAM) formats have become the standard method of representation of
nucleotide sequence alignments for next-generation sequencing data. SAM/BAM
files usually contain information from tens to hundreds of millions of reads.
Often, the sequencing technology, protocol and/or the selected mapping algorithm
introduce some unwanted biases in these data. The systematic detection of such
biases is a non-trivial task that is crucial to drive appropriate downstream
analyses.Results: We have developed Qualimap, a Java application that supports
user-friendly quality control of mapping data, by considering sequence features
and their genomic properties. Qualimap takes sequence alignment data and
provides graphical and statistical analyses for the evaluation of data. Such
quality-control data are vital for highlighting problems in the sequencing
and/or mapping processes, which must be addressed prior to further analyses.
Availability: Qualimap is freely available from http://www.qualimap.org.
Contact:aconesa@cipf.esSupplementary Information:Supplementary data are
available at Bioinformatics online.},
author = {Garc{\'{i}}a-Alcalde, Fernando and Okonechnikov, Konstantin and
Carbonell, Jos{\'{e}} and Cruz, Luis M and G{\"{o}}tz, Stefan and Tarazona,
Sonia and Dopazo, Joaqu{\'{i}}n and Meyer, Thomas F and Conesa, Ana},
doi = {10.1093/bioinformatics/bts503},
issn = {1367-4803},
journal = {Bioinformatics},
month = {aug},
number = {20},
pages = {2678--2679},
title = {{Qualimap: evaluating next-generation sequencing alignment data}},
url = {https://doi.org/10.1093/bioinformatics/bts503},
volume = {28},
year = {2012}
}
\end{verbatim}

Okonechnikov, K., Conesa, A. and García-Alcalde, F. (2015) Qualimap 2: advanced multi-sample quality control for high-throughput sequencing data, Bioinformatics, 32(2), pp. 292–294. doi: 10.1093/bioinformatics/btv566.

\begin{verbatim}
@article{Qualimap2,
abstract = {Motivation: Detection of random errors and systematic biases is a
crucial step of a robust pipeline for processing high-throughput sequencing
(HTS) data. Bioinformatics software tools capable of performing this task are
available, either for general analysis of HTS data or targeted to a specific
sequencing technology. However, most of the existing QC instruments only allow
processing of one sample at a time.Results: Qualimap 2 represents a next step
in the QC analysis of HTS data. Along with comprehensive single-sample analysis
of alignment data, it includes new modes that allow simultaneous processing and
comparison of multiple samples. As with the first version, the new features are
available via both graphical and command line interface. Additionally, it
includes a large number of improvements proposed by the user community.
Availability and implementation: The implementation of the software along with
documentation is freely available at http://www.qualimap.org.
Contact:meyer@mpiib-berlin.mpg.deSupplementary information:Supplementary data
are available at Bioinformatics online.},
author = {Okonechnikov, Konstantin and Conesa, Ana and Garc{\'{i}}a-Alcalde, 
Fernando},
doi = {10.1093/bioinformatics/btv566},
issn = {1367-4803},
journal = {Bioinformatics},
month = {oct},
number = {2},
pages = {292--294},
title = {{Qualimap 2: advanced multi-sample quality control for high-throughput 
sequencing data}},
url = {https://doi.org/10.1093/bioinformatics/btv566},
volume = {32},
year = {2015}
}
\end{verbatim}

\section{FastQC}

Conda: \url{https://anaconda.org/bioconda/fastqc} with GPL 3

Official: \url{http://www.bioinformatics.babraham.ac.uk/projects/fastqc/} with GPL 3

GitHub: \url{https://github.com/s-andrews/FastQC} with GPL 2

Java 95.5\% HTML 3.1\% Perl 1.2\% Other 0.2\%

Missing

\section{CutAdapt}

Conda: \url{https://anaconda.org/bioconda/cutadapt}

GitHub: \url{https://github.com/marcelm/cutadapt}

Official: \url{https://cutadapt.readthedocs.io/en/stable/}

License: MIT

Python 99.4\% Shell 0.6\%



Martin, M. (2011). Cutadapt removes adapter sequences from high-throughput sequencing reads. EMBnet.Journal, 17(1), 10. \url{https://doi.org/10.14806/ej.17.1.200}

\begin{verbatim}
@article{CutAdapt,
abstract = {When small RNA is sequenced on current sequencing machines, the
resulting reads are usually longer than the RNA and therefore contain parts
of the 3' adapter. That adapter must be found and removed error-tolerantly
from each read before read mapping. Previous solutions are either hard to
use or do not offer required features, in particular support for color space
data. As an easy to use alternative, we developed the command-line tool
cutadapt, which supports 454, Illumina and SOLiD (color space) data, offers
two adapter trimming algorithms, and has other useful features.   Cutadapt,
including its MIT-licensed source code, is available for download at
http://code.google.com/p/cutadapt/},
author = {Martin, Marcel},
doi = {10.14806/ej.17.1.200},
journal = {EMBnet.journal},
month = {may},
number = {1},
pages = {10},
publisher = {EMBnet Stichting},
title = {{Cutadapt removes adapter sequences from high-throughput sequencing
reads}},
volume = {17},
year = {2011}
}
\end{verbatim}

\section{Trimmomatic}

Official: \url{http://www.usadellab.org/cms/index.php?page=trimmomatic}

GitHub: \url{https://github.com/timflutre/trimmomatic}

Conda: \url{https://anaconda.org/bioconda/trimmomatic}

License: GPL 3

Java 99.2\% Makefile 0.8\%



Bolger, A. M., Lohse, M., \& Usadel, B. (2014). Trimmomatic: A flexible trimmer for Illumina sequence data. Bioinformatics, 30(15), 2114–2120.\url{https://doi.org/10.1093/bioinformatics/btu170}

\begin{verbatim}
@article{Trimmomatic,
abstract = {Motivation: Although many next-generation sequencing (NGS) read
preprocessing tools already existed, we could not find any tool or
combination of tools that met our requirements in terms of flexibility,
correct handling of paired-end data and high performance. We have developed
Trimmomatic as a more flexible and efficient preprocessing tool, which could
correctly handle paired-end data. Results: The value of NGS read
preprocessing is demonstrated for both reference-based and reference-free
tasks. Trimmomatic is shown to produce output that is at least competitive
with, and in many cases superior to, that produced by other tools, in all
scenarios tested. Availability and implementation: Trimmomatic is licensed
under GPL V3. It is cross-platform (Java 1.5+ required) and available at
http://www.usadellab.org/cms/index.php?page= trimmomatic. {\textcopyright}
The Author 2014.},
author = {Bolger, Anthony M. and Lohse, Marc and Usadel, Bjoern},
doi = {10.1093/bioinformatics/btu170},
issn = {14602059},
journal = {Bioinformatics},
month = {aug},
number = {15},
pages = {2114--2120},
pmid = {24695404},
publisher = {Oxford University Press},
title = {{Trimmomatic: A flexible trimmer for Illumina sequence data}},
volume = {30},
year = {2014}
}
\end{verbatim}

\part{Aligner}

\section{HiSat2}

Conda: \url{https://anaconda.org/bioconda/hisat2}

Official: \url{https://daehwankimlab.github.io/hisat2/}

GitHub: \url{https://github.com/DaehwanKimLab/hisat2}

License: GPL 3

C++ 57.8\% C 16.9\% Python 15.9\% Perl 6.2\% Shell 1.3\% HTML 1.3\% Other 0.6\%



Langmead, B., \& Salzberg, S. L. (2012). Fast gapped-read alignment with Bowtie 2. Nature Methods, 9(4), 357–359. \url{https://doi.org/10.1038/nmeth.1923}


\begin{verbatim}
@article{HiSat2,
abstract = {As the rate of sequencing increases, greater throughput is
demanded from read aligners. The full-text minute index is often used to
make alignment very fast and memory-efficient, but the approach is
ill-suited to finding longer, gapped alignments. Bowtie 2 combines the
strengths of the full-text minute index with the flexibility and speed of
hardware-accelerated dynamic programming algorithms to achieve a combination
of high speed, sensitivity and accuracy. {\textcopyright} 2012 Nature
America, Inc. All rights reserved.},
author = {Langmead, Ben and Salzberg, Steven L.},
doi = {10.1038/nmeth.1923},
issn = {15487091},
journal = {Nature Methods},
month = {apr},
number = {4},
pages = {357--359},
pmid = {22388286},
title = {{Fast gapped-read alignment with Bowtie 2}},
volume = {9},
year = {2012}
}
\end{verbatim}

\section{BWA}

Conda: \url{https://anaconda.org/bioconda/bwa}

GitHub: \url{https://github.com/lh3/bwa}

Official: \url{http://bio-bwa.sourceforge.net/}

SourceForge: \url{https://sourceforge.net/projects/bio-bwa/files/?source=navbar}

License: GPL 3

C 85.7\% JavaScript 5.8\% Roff 4.3\% C++ 1.8\% Perl 1.3\% Shell 0.6\% Makefile 0.5\%



\textbf{BWA-backtrack}

Li, H., \& Durbin, R. (2009). Fast and accurate short read alignment with Burrows-Wheeler transform. Bioinformatics, 25(14), 1754–1760. \url{https://doi.org/10.1093/bioinformatics/btp324}

\begin{verbatim}
@article{BWA1,
abstract = {Motivation: The enormous amount of short reads generated by the
new DNA sequencing technologies call for the development of fast and accurate
read alignment programs. A first generation of hash table-based methods has been
developed, including MAQ, which is accurate, feature rich and fast enough to
align short reads from a single individual. However, MAQ does not support gapped
alignment for single-end reads, which makes it unsuitable for alignment of
longer reads where indels may occur frequently. The speed of MAQ is also a
concern when the alignment is scaled up to the resequencing of hundreds of
individuals. Results: We implemented Burrows-Wheeler Alignment tool (BWA), a new
read alignment package that is based on backward search with Burrows-Wheeler
Transform (BWT), to efficiently align short sequencing reads against a large
reference sequence such as the human genome, allowing mismatches and gaps. BWA
supports both base space reads, e.g. from Illumina sequencing machines, and
color space reads from AB SOLiD machines. Evaluations on both simulated and real
data suggest that BWA is 10-20X faster than MAQ, while achieving similar
accuracy. In addition, BWA outputs alignment in the new standard SAM
(Sequence Alignment/Map) format. Variant calling and other downstream
analyses after the alignment can be achieved with the open source SAMtools
software package. {\textcopyright} 2009 The Author(s).},
author = {Li, Heng and Durbin, Richard},
doi = {10.1093/bioinformatics/btp324},
issn = {13674803},
journal = {Bioinformatics},
month = {jul},
number = {14},
pages = {1754--1760},
pmid = {19451168},
title = {{Fast and accurate short read alignment with Burrows-Wheeler transform}},
volume = {25},
year = {2009}
}
\end{verbatim}

\textbf{BWA-SW}

Li, H., \& Durbin, R. (2010). Fast and accurate long-read alignment with Burrows-Wheeler transform. Bioinformatics, 26(5), 589–595. \url{https://doi.org/10.1093/bioinformatics/btp698}

\begin{verbatim}
@article{BWA2,
abstract = {Motivation: Many programs for aligning short sequencing reads to
a reference genome have been developed in the last 2 years. Most of them are
very efficient for short reads but inefficient or not applicable for reads
{\textgreater}200 bp because the algorithms are heavily and specifically
tuned for short queries with low sequencing error rate. However, some
sequencing platforms already produce longer reads and others are
expected to become available soon. For longer reads, hashingbased
software such as BLAT and SSAHA2 remain the only choices. Nonetheless,
these methods are substantially slower than short-read aligners in terms
of aligned bases per unit time. Results: We designed and implemented a
new algorithm, Burrows-Wheeler Aligner's Smith-Waterman Alignment
(BWA-SW), to align long sequences up to 1Mb against a large sequence
database (e.g. the human genome) with a few gigabytes of memory. The
algorithm is as accurate as SSAHA2, more accurate than BLAT, and is
several to tens of times faster than both. Availability:
http://bio-bwa.sourceforge.net. Contact: rd@sanger.ac.uk
{\textcopyright} The Author(s) 2010. Published by Oxford University
Press.},
author = {Li, Heng and Durbin, Richard},
doi = {10.1093/bioinformatics/btp698},
issn = {13674803},
journal = {Bioinformatics},
month = {jan},
number = {5},
pages = {589--595},
pmid = {20080505},
title = {{Fast and accurate long-read alignment with Burrows-Wheeler transform}},
volume = {26},
year = {2010}
}
\end{verbatim}

\textbf{BWA-MEM}

Li, H. (2013) Aligning sequence reads, clone sequences and assembly contigs with BWA-MEM. Available at: \url{http://arxiv.org/abs/1303.3997} (Accessed: 10 April 2020).

\begin{verbatim}
@article{Li2013,
abstract = {Summary: BWA-MEM is a new alignment algorithm for aligning sequence
reads or long query sequences against a large reference genome such as human. It
automatically chooses between local and end-to-end alignments, supports
paired-end reads and performs chimeric alignment. The algorithm is robust to
sequencing errors and applicable to a wide range of sequence lengths from 70bp
to a few megabases. For mapping 100bp sequences, BWA-MEM shows better
performance than several state-of-art read aligners to date. Availability and
implementation: BWA-MEM is implemented as a component of BWA, which is available
at http://github.com/lh3/bwa. Contact: hengli@broadinstitute.org},
archivePrefix = {arXiv},
arxivId = {1303.3997},
author = {Li, Heng},
eprint = {1303.3997},
month = {mar},
title = {{Aligning sequence reads, clone sequences and assembly contigs with
BWA-MEM}},
url = {http://arxiv.org/abs/1303.3997},
year = {2013}
}
\end{verbatim}

\section{SubRead}

Conda: \url{https://anaconda.org/bioconda/subread}

SourceForge: \url{http://subread.sourceforge.net/}

Official: \url{http://subread.sourceforge.net/}

License: GPL 3



Liao, Y., Smyth, G. K., \& Shi, W. (2013). The Subread aligner: Fast, accurate and scalable read mapping by seed-and-vote. Nucleic Acids Research, 41(10). \url{https://doi.org/10.1093/nar/gkt214}


\begin{verbatim}
@article{SubRead,
abstract = {Read alignment is an ongoing challenge for the analysis of data from
sequencing technologies. This article proposes an elegantly simple multi-seed
strategy, called seed-and-vote, for mapping reads to a reference genome. The new
strategy chooses the mapped genomic location for the read directly from the
seeds. It uses a relatively large number of short seeds (called subreads)
extracted from each read and allows all the seeds to vote on the optimal
location. When the read length is {\textless}160 bp, overlapping subreads are
used. More conventional alignment algorithms are then used to fill in detailed
mismatch and indel information between the subreads that make up the winning
voting block. The strategy is fast because the overall genomic location has
already been chosen before the detailed alignment is done. It is sensitive
because no individual subread is required to map exactly, nor are individual
subreads constrained to map close by other subreads. It is accurate because the
final location must be supported by several different subreads. The strategy
extends easily to find exon junctions, by locating reads that contain sets of
subreads mapping to different exons of the same gene. It scales up efficiently
for longer reads. {\textcopyright} 2013 The Author(s) 2013.},
author = {Liao, Yang and Smyth, Gordon K. and Shi, Wei},
doi = {10.1093/nar/gkt214},
issn = {03051048},
journal = {Nucleic Acids Research},
month = {may},
number = {10},
title = {{The Subread aligner: Fast, accurate and scalable read mapping by
seed-and-vote}},
volume = {41},
year = {2013}
}
\end{verbatim}


Liao, Y., Smyth, G. K. and Shi, W. (2019) The R package Rsubread is easier, faster, cheaper and better for alignment and quantification of RNA sequencing reads, Nucleic acids research. NLM (Medline), 47(8), p. e47. doi: 10.1093/nar/gkz114.

\begin{verbatim}
@article{Liao2019,
abstract = {We present Rsubread, a Bioconductor software package that provides
high-performance alignment and read counting functions for RNA-seq reads.
Rsubread is based on the successful Subread suite with the added ease-of-use of
the R programming environment, creating a matrix of read counts directly as an R
object ready for downstream analysis. It integrates read mapping and
quantification in a single package and has no software dependencies other than R
itself. We demonstrate Rsubread's ability to detect exon-exon junctions de novo
and to quantify expression at the level of either genes, exons or exon
junctions. The resulting read counts can be input directly into a wide range of
downstream statistical analyses using other Bioconductor packages. Using SEQC
data and simulations, we compare Rsubread to TopHat2, STAR and HTSeq as well as
to counting functions in the Bioconductor infrastructure packages. We consider
the performance of these tools on the combined quantification task starting from
raw sequence reads through to summary counts, and in particular evaluate the
performance of different combinations of alignment and counting algorithms. We
show that Rsubread is faster and uses less memory than competitor tools and
produces read count summaries that more accurately correlate with true values.},
author = {Liao, Yang and Smyth, Gordon K. and Shi, Wei},
doi = {10.1093/nar/gkz114},
issn = {13624962},
journal = {Nucleic acids research},
month = {may},
number = {8},
pages = {e47},
pmid = {30783653},
publisher = {NLM (Medline)},
title = {{The R package Rsubread is easier, faster, cheaper and better for
alignment and quantification of RNA sequencing reads}},
volume = {47},
year = {2019}
}
\end{verbatim}

Liao, Y., Smyth, G. K. and Shi, W. (2014) FeatureCounts: An efficient general purpose program for assigning sequence reads to genomic features, Bioinformatics. Oxford University Press, 30(7), pp. 923–930. doi: 10.1093/bioinformatics/btt656.

\begin{verbatim}
@article{Liao2014,
abstract = {Motivation: Next-generation sequencing technologies generate
millions of short sequence reads, which are usually aligned to a reference
genome. In many applications, the key information required for downstream
analysis is the number of reads mapping to each genomic feature, for example to
each exon or each gene. The process of counting reads is called read
summarization. Read summarization is required for a great variety of genomic
analyses but has so far received relatively little attention in the literature.
Results: We present featureCounts, a read summarization program suitable for
counting reads generated from either RNA or genomic DNA sequencing experiments.
featureCounts implements highly efficient chromosome hashing and feature
blocking techniques. It is considerably faster than existing methods
(by an order of magnitude for gene-level summarization) and requires far less
computer memory. It works with either single or paired-end reads and provides a
wide range of options appropriate for different sequencing applications.
{\textcopyright} 2013 The Author 2013. Published by Oxford University Press.
All rights reserved.},
archivePrefix = {arXiv},
arxivId = {1305.3347},
author = {Liao, Yang and Smyth, Gordon K. and Shi, Wei},
doi = {10.1093/bioinformatics/btt656},
eprint = {1305.3347},
issn = {14602059},
journal = {Bioinformatics},
month = {apr},
number = {7},
pages = {923--930},
pmid = {24227677},
publisher = {Oxford University Press},
title = {{FeatureCounts: An efficient general purpose program for assigning
sequence reads to genomic features}},
volume = {30},
year = {2014}
}
\end{verbatim}

\section{Bowtie}

Official: \url{http://bowtie-bio.sourceforge.net/index.shtml}

GitHub: \url{https://github.com/BenLangmead/bowtie}

Conda: \url{https://anaconda.org/bioconda/bowtie}

License: Artistic License 2.0

C++ 91.1\% Perl 6.2\% Shell 0.9\% C 0.7\% Python 0.6\% Makefile 0.4\% Objective-C 0.1\%



Langmead, B., Trapnell, C., Pop, M., \& Salzberg, S. L. (2009). Ultrafast and memory-efficient alignment of short DNA sequences to the human genome. Genome Biology, 10(3), R25. \url{https://doi.org/10.1186/gb-2009-10-3-r25}Langmead, B., \& Salzberg, S. L. (2012).

\begin{verbatim}
@article{Bowtie,
author = {Langmead, Ben and Trapnell, Cole and Pop, Mihai and Salzberg, Steven L},
doi = {10.1186/gb-2009-10-3-r25},
issn = {1465-6906},
journal = {Genome Biology},
number = {3},
pages = {R25},
title = {{Ultrafast and memory-efficient alignment of short DNA sequences to the
human genome}},
url = {http://genomebiology.biomedcentral.com/articles/10.1186/gb-2009-10-3-r25},
volume = {10},
year = {2009}
}
\end{verbatim}

\section{Bowtie2}

Conda: \url{https://anaconda.org/bioconda/bowtie2}

Official: \url{http://bowtie-bio.sourceforge.net/bowtie2/index.shtml}

GitHub: \url{https://github.com/BenLangmead/bowtie2}

License: GPL 3

C++ 80.7\% Perl 14.5\% Python 1.5\% C 1.3\% Shell 1.1\% Makefile 0.6\% CMake 0.3\%



Fast gapped-read alignment with Bowtie 2. Nature Methods, 9(4), 357–359. \url{https://doi.org/10.1038/nmeth.1923}

\begin{verbatim}
@article{Bowtie2,
abstract = {As the rate of sequencing increases, greater throughput is demanded
from read aligners. The full-text minute index is often used to make alignment
very fast and memory-efficient, but the approach is ill-suited to finding
longer, gapped alignments. Bowtie 2 combines the strengths of the full-text
minute index with the flexibility and speed of hardware-accelerated dynamic
programming algorithms to achieve a combination of high speed, sensitivity and
accuracy. {\textcopyright} 2012 Nature America, Inc. All rights reserved.},
author = {Langmead, Ben and Salzberg, Steven L.},
doi = {10.1038/nmeth.1923},
issn = {15487091},
journal = {Nature Methods},
month = {apr},
number = {4},
pages = {357--359},
pmid = {22388286},
title = {{Fast gapped-read alignment with Bowtie 2}},
volume = {9},
year = {2012}
}
\end{verbatim}

\section{TopHat}

Official: \url{http://ccb.jhu.edu/software/tophat/index.shtml}

GitHub: \url{https://github.com/infphilo/tophat}

Conda: \url{https://anaconda.org/bioconda/tophat}

License: BSL 1.0

C++ 84.8\% C 10.0\% Python 3.8\% Perl 0.6\% Makefile 0.3\% M4 0.2\% Other 0.3\%



Trapnell, C., Pachter, L. and Salzberg, S. L. (2009) TopHat: discovering splice junctions with RNA-Seq., Bioinformatics (Oxford, England), 25(9), pp. 1105–1111. doi: 10.1093/bioinformatics/btp120.

\begin{verbatim}
@article{Trapnell2009,
abstract = {MOTIVATION: A new protocol for sequencing the messenger RNA in a
cell, known as  RNA-Seq, generates millions of short sequence fragments in a
single run. These fragments, or 'reads', can be used to measure levels of gene
expression and to identify novel splice variants of genes. However, current
software for aligning RNA-Seq data to a genome relies on known splice junctions
and cannot identify novel ones. TopHat is an efficient read-mapping algorithm
designed to align reads from an RNA-Seq experiment to a reference genome without
relying on known splice sites. RESULTS: We mapped the RNA-Seq reads from a
recent mammalian RNA-Seq experiment and recovered more than 72{\%} of the splice
junctions reported by the annotation-based software from that study, along with
nearly 20,000 previously unreported junctions. The TopHat pipeline is much
faster than previous systems, mapping nearly 2.2 million reads per CPU hour,
which is sufficient to process an entire RNA-Seq experiment in less than a day
on a standard desktop computer. We describe several challenges unique to ab
initio splice site discovery from RNA-Seq reads that will require further
algorithm development. AVAILABILITY: TopHat is free, open-source software
available from http://tophat.cbcb.umd.edu. SUPPLEMENTARY INFORMATION:
Supplementary data are available at Bioinformatics online.},
author = {Trapnell, Cole and Pachter, Lior and Salzberg, Steven L},
doi = {10.1093/bioinformatics/btp120},
issn = {1367-4811 (Electronic)},
journal = {Bioinformatics (Oxford, England)},
keywords = {Algorithms,Gene Expression Profiling,Models, Genetic,RNA Splicing,
RNA, Messenger,Sequence Alignment,Sequence Analysis, RNA,Software,genetics,methods},
language = {eng},
month = {may},
number = {9},
pages = {1105--1111},
pmid = {19289445},
title = {{TopHat: discovering splice junctions with RNA-Seq.}},
volume = {25},
year = {2009}
}

\end{verbatim}

Langmead, B., Trapnell, C., Pop, M. and Salzberg, S. L. (2009) Ultrafast and memory-efficient alignment of short DNA sequences to the human genome, Genome Biology. BioMed Central, 10(3), p. R25. doi: 10.1186/gb-2009-10-3-r25.

\begin{verbatim}
@article{Langmead2009,
abstract = {Bowtie is an ultrafast, memory-efficient alignment program for
aligning short DNA sequence reads to large genomes. For the human genome,
Burrows-Wheeler indexing allows Bowtie to align more than 25 million reads per
CPU hour with a memory footprint of approximately 1.3 gigabytes. Bowtie extends
previous Burrows-Wheeler techniques with a novel quality-aware backtracking
algorithm that permits mismatches. Multiple processor cores can be used
simultaneously to achieve even greater alignment speeds. Bowtie is open source
http://bowtie.cbcb.umd.edu. {\textcopyright} 2009 Langmead et al.; licensee
BioMed Central Ltd.},
author = {Langmead, Ben and Trapnell, Cole and Pop, Mihai and Salzberg, Steven 
L.},
doi = {10.1186/gb-2009-10-3-r25},
issn = {14747596},
journal = {Genome Biology},
keywords = {Animal Genetics and Genomics,Bioinformatics,Evolutionary Biology,
Human Genetics,Microbial Genetics and Genomics,Plant Genetics and Genomics},
month = {mar},
number = {3},
pages = {R25},
pmid = {19261174},
publisher = {BioMed Central},
title = {{Ultrafast and memory-efficient alignment of short DNA sequences to the
human genome}},
url = {http://genomebiology.biomedcentral.com/articles/10.1186/gb-2009-10-3-r25},
volume = {10},
year = {2009}
}

\end{verbatim}

Kim, D., Pertea, G., Trapnell, C., Pimentel, H., Kelley, R. and Salzberg, S. L. (2013) TopHat2: Accurate alignment of transcriptomes in the presence of insertions, deletions and gene fusions, Genome Biology. BioMed Central, 14(4), p. R36. doi: 10.1186/gb-2013-14-4-r36.

\begin{verbatim}
@article{Kim2013,
abstract = {TopHat is a popular spliced aligner for RNA-sequence (RNA-seq)
experiments. In this paper, we describe TopHat2, which incorporates many
significant enhancements to TopHat. TopHat2 can align reads of various lengths
produced by the latest sequencing technologies, while allowing for
variable-length indels with respect to the reference genome. In addition to de
novo spliced alignment, TopHat2 can align reads across fusion breaks, which can
occur after genomic translocations. TopHat2 combines the ability to identify
novel splice sites with direct mapping to known transcripts, producing sensitive
and accurate alignments, even for highly repetitive genomes or in the presence
of pseudogenes. TopHat2 is available at http://ccb.jhu.edu/software/tophat.
{\textcopyright} 2013 Kim et al.; licensee BioMed Central Ltd.},
author = {Kim, Daehwan and Pertea, Geo and Trapnell, Cole and Pimentel, Harold
and Kelley, Ryan and Salzberg, Steven L.},
doi = {10.1186/gb-2013-14-4-r36},
issn = {1474760X},
journal = {Genome Biology},
keywords = {Animal Genetics and Genomics,Bioinformatics,Evolutionary Biology,
Human Genetics,Microbial Genetics and Genomics,Plant Genetics and Genomics},
month = {apr},
number = {4},
pages = {R36},
pmid = {23618408},
publisher = {BioMed Central},
title = {{TopHat2: Accurate alignment of transcriptomes in the presence of
insertions, deletions and gene fusions}},
url = {http://genomebiology.biomedcentral.com/articles/10.1186/gb-2013-14-4-r36},
volume = {14},
year = {2013}
}

\end{verbatim}

Kim, D. and Salzberg, S. L. (2011) TopHat-Fusion: An algorithm for discovery of novel fusion transcripts, Genome Biology. BioMed Central, 12(8), p. R72. doi: 10.1186/gb-2011-12-8-r72.

\begin{verbatim}
@article{Kim2011,
abstract = {TopHat-Fusion is an algorithm designed to discover transcripts
representing fusion gene products, which result from the breakage and re-joining
of two different chromosomes, or from rearrangements within a chromosome.
TopHat-Fusion is an enhanced version of TopHat, an efficient program that aligns
RNA-seq reads without relying on existing annotation. Because it is independent
of gene annotation, TopHat-Fusion can discover fusion products deriving from
known genes, unknown genes and unannotated splice variants of known genes. Using
RNA-seq data from breast and prostate cancer cell lines, we detected both
previously reported and novel fusions with solid supporting evidence.
{\textcopyright} 2011 Kim and Salzberg; licensee BioMed Central Ltd.},
author = {Kim, Daehwan and Salzberg, Steven L.},
doi = {10.1186/gb-2011-12-8-r72},
issn = {14747596},
journal = {Genome Biology},
keywords = {Animal Genetics and Genomics,Bioinformatics,Evolutionary Biology,Human
Genetics,Microbial Genetics and Genomics,Plant Genetics and Genomics},
month = {aug},
number = {8},
pages = {R72},
publisher = {BioMed Central},
title = {{TopHat-Fusion: An algorithm for discovery of novel fusion transcripts}},
url = {http://genomebiology.biomedcentral.com/articles/10.1186/gb-2011-12-8-r72},
volume = {12},
year = {2011}
}

\end{verbatim}

\part{SAM/BAM/CRAM File Processor}

\section{Sambamba}

Conda: \url{https://anaconda.org/bioconda/sambamba}

Official: \url{http://www.sambamba.org/}

GitHub: \url{https://github.com/biod/sambamba}

License: GPL 2

D 82.0\% Shell 7.7\% Roff 6.0\% Python 1.6\% Ruby 1.1\% Meson 0.8\% Makefile 0.8\%


Tarasov, A., Vilella, A. J., Cuppen, E., Nijman, I. J., \& Prins, P. (2015). Sambamba: fast processing of NGS alignment formats. Bioinformatics, 31(12), 2032–2034. \url{https://doi.org/10.1093/bioinformatics/btv098}

\begin{verbatim}
@article{Sambamba,
author = {Tarasov, Artem and Vilella, Albert J. and Cuppen, Edwin and Nijman,
Isaac J. and Prins, Pjotr},
doi = {10.1093/bioinformatics/btv098},
issn = {1367-4803},
journal = {Bioinformatics},
month = {jun},
number = {12},
pages = {2032--2034},
title = {{Sambamba: fast processing of NGS alignment formats}},
url = {https://academic.oup.com/bioinformatics/article-lookup/doi/10.1093
/bioinformatics/btv098},
volume = {31},
year = {2015}
}
\end{verbatim}


\section{Samtools}

Official: \url{http://www.htslib.org/}

SourceForge: \url{http://samtools.sourceforge.net/}

Conda: \url{https://anaconda.org/bioconda/samtools}

GitHub: \url{https://github.com/samtools/samtools}

License: The MIT/Expat License

C 73.3\% Perl 18.8\% M4 2.7\% Lua 1.8\% Shell 1.4\% Makefile 1.0\% Other 1.0\%




\textbf{VCF Format:}

Danecek, P., Auton, A., Abecasis, G., Albers, C. A., Banks, E., DePristo, M. A., … Durbin, R. (2011). The variant call format and VCFtools. Bioinformatics, 27(15), 2156–2158. \url{https://doi.org/10.1093/bioinformatics/btr330}

\begin{verbatim}
@article{VCFTools,
abstract = {Summary: The variant call format (VCF) is a generic format for
storing DNA polymorphism data such as SNPs, insertions, deletions and structural
variants, together with rich annotations. VCF is usually stored in a compressed
manner and can be indexed for fast data retrieval of variants from a range of
positions on the reference genome. The format was developed for the 1000 Genomes
Project, and has also been adopted by other projects such as UK10K, dbSNP and
the NHLBI Exome Project. VCFtools is a software suite that implements various
utilities for processing VCF files, including validation, merging, comparing and
also provides a general Perl API. {\textcopyright} The Author(s) 2011. Published
by Oxford University Press.},
author = {Danecek, Petr and Auton, Adam and Abecasis, Goncalo and Albers,
Cornelis A. and Banks, Eric and DePristo, Mark A. and Handsaker, Robert E. and
Lunter, Gerton and Marth, Gabor T. and Sherry, Stephen T. and McVean, Gilean and
Durbin, Richard},
doi = {10.1093/bioinformatics/btr330},
issn = {13674803},
journal = {Bioinformatics},
month = {aug},
number = {15},
pages = {2156--2158},
pmid = {21653522},
title = {{The variant call format and VCFtools}},
volume = {27},
year = {2011}
}
\end{verbatim}

\textbf{SAM \& BAM Format:}


Cock, P. J. A., Bonfield, J. K., Chevreux, B., \& Li, H. (2015). SAM/BAM format v1.5 extensions for de novo assemblies. BioRxiv, 020024. \url{https://doi.org/10.1101/020024}

\begin{verbatim}
@article{SAM,
abstract = {Summary: The Sequence Alignment/Map (SAM) format is a generic
alignment format for storing read alignments against reference sequences,
supporting short and long reads (up to 128 Mbp) produced by different sequencing
platforms. It is flexible in style, compact in size, efficient in random access
and is the format in which alignments from the 1000 Genomes Project are
released. SAM tools implements various utilities for post-processing alignments
in the SAM format, such as indexing, variant caller and alignment viewer, and
thus provides universal tools for processing read alignments. {\textcopyright}
2009 The Author(s).},
author = {Li, Heng and Handsaker, Bob and Wysoker, Alec and Fennell, Tim and
Ruan, Jue and Homer, Nils and Marth, Gabor and Abecasis, Goncalo and Durbin,
Richard},
doi = {10.1093/bioinformatics/btp352},
issn = {13674803},
journal = {Bioinformatics},
month = {aug},
number = {16},
pages = {2078--2079},
pmid = {19505943},
title = {{The Sequence Alignment/Map format and SAMtools}},
volume = {25},
year = {2009}
}
\end{verbatim}

Li, H., Handsaker, B., Wysoker, A., Fennell, T., Ruan, J., Homer, N., … Durbin, R. (2009). The Sequence Alignment/Map format and SAMtools. Bioinformatics, 25(16), 2078–2079. \url{https://doi.org/10.1093/bioinformatics/btp352}

\begin{verbatim}
@article{SAM2,
abstract = {Summary: The plain text Sequence Alignment/Map (SAM) file format and
its companion binary form (BAM) are a generic alignment format for storing read
alignments against reference sequences (and unmapped reads) together with
structured meta-data (Li et al., 2009). Driven by the needs of the 1000 Genomes
Project which sequenced many individual human genomes, early SAM/BAM usage
focused on pairwise alignments of reads to a reference. However, through the
CIGAR P operator multiple sequence alignments can also be preserved. Herein we
describe clarifications and additions in version 1.5 of the specification to
facilitate storing de novo sequence alignments: Padded reference sequences (with
gap characters), annotation of reads or regions of the reference, and the option
of embedding the reference sequence within the file. Availability: The latest
public release of the specification is at
http://samtools.sourceforge.net/SAM1.pdf, with in development drafts at
https://github.com/samtools/hts-specs/ under version control.},
author = {Cock, Peter J. A. and Bonfield, James K and Chevreux, Bastien and Li, Heng},
doi = {10.1101/020024},
journal = {bioRxiv},
pages = {020024},
title = {{SAM/BAM format v1.5 extensions for de novo assemblies}},
url = {http://biorxiv.org/content/early/2015/05/29/020024.abstract
http://biorxiv.org/lookup/doi/10.1101/020024},
year = {2015}
}
\end{verbatim}

\textbf{CRAM Format:}

Fritz, M. H. Y., Leinonen, R., Cochrane, G., \& Birney, E. (2011). Efficient storage of high throughput DNA sequencing data using reference-based compression. Genome Research, 21(5), 734–740. \url{https://doi.org/10.1101/gr.114819.110}

\begin{verbatim}
@article{CRAM,
abstract = {Data storage costs have become an appreciable proportion of total
cost in the creation and analysis of DNA sequence data. Of particular concern is
that the rate of increase in DNA sequencing is significantly outstripping the
rate of increase in disk storage capacity. In this paper we present a new
reference-based compression method that efficiently compresses DNA sequences for
storage. Our approach works for resequencing experiments that target
well-studied genomes. We align new sequences to a reference genome and then
encode the differences between the new sequence and the reference genome for
storage. Our compression method is most efficient when we allow controlled loss
of data in the saving of quality information and unaligned sequences. With this
new compression method we observe exponential efficiency gains as read lengths
increase, and the magnitude of this efficiency gain can be controlled by
changing the amount of quality information stored. Our compression method is
tunable: The storage of quality scores and unaligned sequences may be adjusted
for different experiments to conserve information or to minimize storage costs,
and provides one opportunity to address the threat that increasing DNA sequence
volumes will overcome our ability to store the sequences. {\textcopyright} 2011
by Cold Spring Harbor Laboratory Press.},
author = {Fritz, Markus Hsi Yang and Leinonen, Rasko and Cochrane, Guy and
Birney, Ewan},
doi = {10.1101/gr.114819.110},
issn = {10889051},
journal = {Genome Research},
month = {may},
number = {5},
pages = {734--740},
title = {{Efficient storage of high throughput DNA sequencing data using
reference-based compression}},
volume = {21},
year = {2011}
}
\end{verbatim}

\textbf{The original mpileup calling algorithm plus mathematical notes (mpileup/bcftools call -c)}

\begin{verbatim}
@article{bcftools1,
abstract = {Motivation: Most existing methods for DNA sequence analysis rely on
accurate sequences or genotypes. However, in applications of the next-generation
sequencing (NGS), accurate genotypes may not be easily obtained (e.g.
multi-sample low-coverage sequencing or somatic mutation discovery). These
applications press for the development of new methods for analyzing sequence
data with uncertainty. Results: We present a statistical framework for calling
SNPs, discovering somatic mutations, inferring population genetical parameters
and performing association tests directly based on sequencing data without
explicit genotyping or linkage-based imputation. On real data, we demonstrate
that our method achieves comparable accuracy to alternative methods for
estimating site allele count, for inferring allele frequency spectrum and for
association mapping. We also highlight the necessity of using symmetric datasets
for finding somatic mutations and confirm that for discovering rare events,
mismapping is frequently the leading source of errors. {\textcopyright} The
Author 2011. Published by Oxford University Press. All rights reserved.},
archivePrefix = {arXiv},
arxivId = {1203.6372},
author = {Li, Heng},
doi = {10.1093/bioinformatics/btr509},
eprint = {1203.6372},
issn = {13674803},
journal = {Bioinformatics},
month = {nov},
number = {21},
pages = {2987--2993},
title = {{A statistical framework for SNP calling, mutation discovery,
association mapping and population genetical parameter estimation from
sequencing data}},
volume = {27},
year = {2011}
}
\end{verbatim}

Li, H. (2011). A statistical framework for SNP calling, mutation discovery, association mapping and population genetical parameter estimation from sequencing data. Bioinformatics, 27(21), 2987–2993. \url{https://doi.org/10.1093/bioinformatics/btr509}

\begin{verbatim}
@article{bcftools2,
author = {Li, Heng},
title = {{Mathematical Notes on SAMtools Algorithms}},
year = {2010}
}
\end{verbatim}

LI, H. (2010). Mathematical Notes on SAMtools Algorithms.

\begin{verbatim}
@article{Li2010,
author = {Li, Heng},
number = {2},
pages = {245--7},
title = {{Mathematical Notes on SAMtools Algorithms}},
volume = {30},
year = {2010}
}
\end{verbatim}

\textbf{Mathematical notes for the updated multiallelic calling model (mpileup/bcftools call -m)}

Petr Danecek, Stephan Schiffels, R. D. (2016). Multiallelic calling model in bcftools (-m). Retrieved January 30, 2020, from \url{http://samtools.github.io/bcftools/call-m.pdf}

\begin{verbatim}
@article{bcftools3,
author = {Petr Danecek, Stephan Schiffels, Richard Durbin},
title = {{Multiallelic calling model in bcftools (-m)}},
year = {2016},
url={http://samtools.github.io/bcftools/call-m.pdf},
urldate = {2020-01-30}
}
\end{verbatim}

\textbf{Hidden Markov model for detecting runs of homozygosity (bcftools roh)}

Narasimhan, V., Danecek, P., Scally, A., Xue, Y., Tyler-Smith, C., \& Durbin, R. (2016). BCFtools/RoH: A hidden Markov model approach for detecting autozygosity from next-generation sequencing data. Bioinformatics, 32(11), 1749–1751. \url{https://doi.org/10.1093/bioinformatics/btw044}

\begin{verbatim}
@article{bcftools4,
abstract = {Summary: Runs of homozygosity (RoHs) are genomic
stretches of a diploid genome that show identical alleles on both chromosomes.
Longer RoHs are unlikely to have arisen by chance but are likely to denote
autozygosity, whereby both copies of the genome descend from the same recent
ancestor. Early tools to detect RoH used genotype array data, but substantially
more information is available from sequencing data. Here, we present and
evaluate BCFtools/RoH, an extension to the BCFtools software package, that
detects regions of autozygosity in sequencing data, in particular exome data,
using a hidden Markov model. By applying it to simulated data and real data from
the 1000 Genomes Project we estimate its accuracy and show that it has higher
sensitivity and specificity than existing methods under a range of sequencing
error rates and levels of autozygosity. Availability and implementation:
BCFtools/RoH and its associated binary/source files are freely available from
https://github.com/samtools/BCFtools. Supplementary information: Supplementary
data are available at Bioinformatics online.},
author = {Narasimhan, Vagheesh and Danecek, Petr and Scally, Aylwyn and Xue,
Yali and Tyler-Smith, Chris and Durbin, Richard},
doi = {10.1093/bioinformatics/btw044},
issn = {14602059},
journal = {Bioinformatics},
month = {jun},
number = {11},
pages = {1749--1751},
pmid = {26826718},
publisher = {Oxford University Press},
title = {{BCFtools/RoH: A hidden Markov model approach for detecting autozygosity
 from next-generation sequencing data}},
volume = {32},
year = {2016}
}
\end{verbatim}

\textbf{Copy number variation/aneuploidy calling from microarray data (bcftools cnv/bcftools polysomy)}


Danecek, P., McCarthy, S. A., Consortium, H. S., \& Durbin, R. (2016). A method for checking genomic integrity in cultured cell lines from snp genotyping data. PLoS ONE, 11(5). \url{https://doi.org/10.1371/journal.pone.0155014}

\begin{verbatim}
@article{bcftools5,
abstract = {Genomic screening for chromosomal abnormalities is an important part
of quality control when establishing and maintaining stem cell lines. We present
a new method for sensitive detection of copy number alterations, aneuploidy, and
contamination in cell lines using genome-wide SNP genotyping data. In contrast
to other methods designed for identifying copy number variations in a single
sample or in a sample composed of a mixture of normal and tumor cells, this new
method is tailored for determining differences between cell lines and the
starting material from which they were derived, which allows us to distinguish
between normal and novel copy number variation. We implemented the method in the
freely available BCFtools package and present results based on induced
pluripotent stem cell lines obtained in the HipSci project.},
author = {Danecek, Petr and McCarthy, Shane A. and Consortium, Hip Sci and
Durbin, Richard},
doi = {10.1371/journal.pone.0155014},
issn = {19326203},
journal = {PLoS ONE},
month = {may},
number = {5},
publisher = {Public Library of Science},
title = {{A method for checking genomic integrity in cultured cell lines from snp
genotyping data}},
volume = {11},
year = {2016}
}
\end{verbatim}

\textbf{Haplotype-aware calling of variant consequences (bcftools csq)}

Danecek, P., \& McCarthy, S. A. (2017). BCFtools/csq: Haplotype-aware variant consequences. Bioinformatics, 33(13), 2037–2039. \url{https://doi.org/10.1093/bioinformatics/btx100}

\begin{verbatim}
@article{bcftools6,
abstract = {Motivation: Prediction of functional variant consequences is an
important part of sequencing pipelines, allowing the categorization and
prioritization of genetic variants for follow up analysis. However, current
predictors analyze variants as isolated events, which can lead to incorrect
predictions when adjacent variants alter the same codon, or when a
frame-shifting indel is followed by a frame-restoring indel. Exploiting known
haplotype information when making consequence predictions can resolve these
issues. Results: BCFtools/csq is a fast program for haplotype-aware consequence
calling which can take into account known phase. Consequence predictions are
changed for 501 of 5019 compound variants found in the 81.7M variants in the
1000 Genomes Project data, with an average of 139 compound variants per
haplotype. Predictions match existing tools when run in localized mode, but the
program is an order of magnitude faster and requires an order of magnitude less
memory. Availability and Implementation: The program is freely available for
commercial and noncommercial use in the BCFtools package which is available for
download from http://samtools. github.io/bcftools.},
author = {Danecek, Petr and McCarthy, Shane A.},
doi = {10.1093/bioinformatics/btx100},
issn = {14602059},
journal = {Bioinformatics},
month = {jul},
number = {13},
pages = {2037--2039},
publisher = {Oxford University Press},
title = {{BCFtools/csq: Haplotype-aware variant consequences}},
volume = {33},
year = {2017}
}
\end{verbatim}

\textbf{Base alignment quality (BAQ) method improve SNP calling around INDELs}


Li, H. (2011). Improving SNP discovery by base alignment quality. Bioinformatics, 27(8), 1157–1158. \url{https://doi.org/10.1093/bioinformatics/btr076}

\begin{verbatim}
@article{bcftools7,
abstract = {Summary: I propose a new application of profile Hidden Markov Models
in the area of SNP discovery from resequencing data, to greatly reduce false SNP
calls caused by misalignments around insertions and deletions (indels). The
central concept is per-Base Alignment Quality, which accurately measures the
probability of a read base being wrongly aligned. The effectiveness of BAQ has
been positively confirmed on large datasets by the 1000 Genomes Project analysis
subgroup. {\textcopyright} The Author 2011. Published by Oxford University
Press. All rights reserved.},
author = {Li, Heng},
doi = {10.1093/bioinformatics/btr076},
issn = {13674803},
journal = {Bioinformatics},
month = {apr},
number = {8},
pages = {1157--1158},
title = {{Improving SNP discovery by base alignment quality}},
volume = {27},
year = {2011}
}
\end{verbatim}

\textbf{Segregation based QC metric originally implemented in SGA}

Durbin, R. (2014). Segregation based metric for variant call QC. Retrieved from \url{http://samtools.github.io/bcftools/rd-SegBias.pdf}

\begin{verbatim}
@article{bcftools8,
author = {Durbin, Richard},
title = {{Segregation based metric for variant call QC}},
url = {http://samtools.github.io/bcftools/rd-SegBias.pdf},
year = {2014}
}
\end{verbatim}


\part{Variant Caller}

\section{VarScan1}

Official: \url{http://varscan.sourceforge.net/}

Java

Koboldt, D. C., Chen, K., Wylie, T., Larson, D. E., McLellan, M. D., Mardis, E. R., Weinstock, G. M., Wilson, R. K. and Ding, L. (2009) VarScan: variant detection in massively parallel sequencing of individual and pooled samples., Bioinformatics (Oxford, England), 25(17), pp. 2283–5. doi: 10.1093/bioinformatics/btp373.

\begin{verbatim}
@article{Koboldt2009,
abstract = {SUMMARY Massively parallel sequencing technologies hold incredible
promise for the study of DNA sequence variation, particularly the identification
of variants affecting human disease. The unprecedented throughput and relatively
short read lengths of Roche/454, Illumina/Solexa, and other platforms have
spurred development of a new generation of sequence alignment algorithms. Yet
detection of sequence variants based on short read alignments remains challenging,
and most currently available tools are limited to a single platform or aligner
type. We present VarScan, an open source tool for variant detection that is
compatible with several short read aligners. We demonstrate VarScan's ability to
detect SNPs and indels with high sensitivity and specificity, in both Roche/454
sequencing of individuals and deep Illumina/Solexa sequencing of pooled samples.},
author = {Koboldt, Daniel C and Chen, Ken and Wylie, Todd and Larson, David E and
McLellan, Michael D and Mardis, Elaine R and Weinstock, George M and Wilson,
Richard K and Ding, Li},
doi = {10.1093/bioinformatics/btp373},
issn = {1367-4811},
journal = {Bioinformatics (Oxford, England)},
month = {sep},
number = {17},
pages = {2283--5},
pmid = {19542151},
title = {{VarScan: variant detection in massively parallel sequencing of individual
and pooled samples.}},
url = {http://www.ncbi.nlm.nih.gov/pubmed/19542151
http://www.pubmedcentral.nih.gov/articlerender.fcgi?artid=PMC2734323},
volume = {25},
year = {2009}
}

\end{verbatim}


\section{VarScan2}


Official: \url{http://dkoboldt.github.io/varscan/}

GitHub: \url{https://github.com/dkoboldt/varscan}

Conda: \url{https://anaconda.org/bioconda/varscan}

License: The Non-Profit Open Software License version 3.0 (NPOSL-3.0)

Java

Koboldt, D. C., Zhang, Q., Larson, D. E., Shen, D., McLellan, M. D., Lin, L., Miller, C. A., Mardis, E. R., Ding, L. and Wilson, R. K. (2012) VarScan 2: Somatic mutation and copy number alteration discovery in cancer by exome sequencing, Genome Research. Cold Spring Harbor Laboratory Press, 22(3), pp. 568–576. doi: 10.1101/gr.129684.111.

\begin{verbatim}
@article{Koboldt2012,
abstract = {Cancer is a disease driven by genetic variation and mutation. Exome
sequencing can be utilized for discovering these variants and mutations across
hundreds of tumors. Here we present an analysis tool, VarScan 2, for the
detection of somatic mutations and copy number alterations (CNAs) in exome data
from tumor-normal pairs. Unlike most current approaches, our algorithm reads
data from both samples simultaneously; a heuristic and statistical algorithm
detects sequence variants and classifies them by somatic status (germline,
somatic, or LOH); while a comparison of normalized read depth delineates relative
copy number changes. We apply these methods to the analysis of exome sequence
data from 151 high-grade ovarian tumors characterized as part of the Cancer
Genome Atlas (TCGA). We validated some 7790 somatic coding mutations, achieving
93{\%} sensitivity and 85{\%} precision for single nucleotide variant (SNV)
detection. Exome-based CNA analysis identified 29 large-scale alterations and
619 focal events per tumor on average. As in our previous analysis of these
data, we observed frequent amplification of oncogenes (e.g., CCNE1, MYC) and
deletion of tumor suppressors (NF1, PTEN, and CDKN2A). We searched for
additional recurrent focal CNAs using the correlation matrix diagonal
segmentation (CMDS) algorithm, which identified 424 significant events affecting
582 genes. Taken together, our results demonstrate the robust performance of
VarScan 2 for somatic mutation and CNA detection and shed new light on the
landscape of genetic alterations in ovarian cancer. {\textcopyright} 2012 by
Cold Spring Harbor Laboratory Press.},
author = {Koboldt, Daniel C. and Zhang, Qunyuan and Larson, David E. and Shen,
Dong and McLellan, Michael D. and Lin, Ling and Miller, Christopher A. and
Mardis, Elaine R. and Ding, Li and Wilson, Richard K.},
doi = {10.1101/gr.129684.111},
issn = {10889051},
journal = {Genome Research},
month = {mar},
number = {3},
pages = {568--576},
publisher = {Cold Spring Harbor Laboratory Press},
title = {{VarScan 2: Somatic mutation and copy number alteration discovery
in cancer by exome sequencing}},
volume = {22},
year = {2012}
}

\end{verbatim}
\section{LoFreq}

SourceForge: \url{https://sourceforge.net/projects/lofreq/}

Conda: \url{https://anaconda.org/bioconda/lofreq}

Official: \url{http://csb5.github.io/lofreq/}

GitHub: \url{https://github.com/CSB5/lofreq}

License: MIT License


C 74.2\% Python 17.8\% Shell 5.3\% M4 2.3\% Other 0.4\%

Wilm, A., Aw, P. P. K., Bertrand, D., Yeo, G. H. T., Ong, S. H., Wong, C. H., … Nagarajan, N. (2012). LoFreq: A sequence-quality aware, ultra-sensitive variant caller for uncovering cell-population heterogeneity from high-throughput sequencing datasets. Nucleic Acids Research, 40(22), 11189–11201. \url{https://doi.org/10.1093/nar/gks918}

\begin{verbatim}
@article{LoFreq,
abstract = {The study of cell-population heterogeneity in a range of biological
systems, from viruses to bacterial isolates to tumor samples, has been
transformed by recent advances in sequencing throughput. While the high-coverage
afforded can be used, in principle, to identify very rare variants in a
population, existing ad hoc approaches frequently fail to distinguish true
variants from sequencing errors. We report a method (LoFreq) that models
sequencing run-specific error rates to accurately call variants occurring in
{\textless}0.05{\%} of a population. Using simulated and real datasets
(viral, bacterial and human), we show that LoFreq has near-perfect specificity,
with significantly improved sensitivity compared with existing methods and can
efficiently analyze deep Illumina sequencing datasets without resorting to
approximations or heuristics. We also present experimental validation for
LoFreq on two different platforms (Fluidigm and Sequenom) and its application
to call rare somatic variants from exome sequencing datasets for gastric cancer.
Source code and executables for LoFreq are freely available at
http://sourceforge.net/projects/ lofreq/. {\textcopyright} 2012 The Author(s).},
author = {Wilm, Andreas and Aw, Pauline Poh Kim and Bertrand, Denis and Yeo,
Grace Hui Ting and Ong, Swee Hoe and Wong, Chang Hua and Khor, Chiea Chuen and
Petric, Rosemary and Hibberd, Martin Lloyd and Nagarajan, Niranjan},
doi = {10.1093/nar/gks918},
issn = {03051048},
journal = {Nucleic Acids Research},
month = {dec},
number = {22},
pages = {11189--11201},
title = {{LoFreq: A sequence-quality aware, ultra-sensitive variant caller for
uncovering cell-population heterogeneity from high-throughput sequencing datasets}},
volume = {40},
year = {2012}
}
\end{verbatim}

\section{MuSE}

Official: \url{https://bioinformatics.mdanderson.org/public-software/muse/}

GitHub: \url{https://github.com/danielfan/MuSE}

Conda: \url{https://anaconda.org/bioconda/muse}

License: GPL 2

C++ 63.0\% C 36.8\% Makefile 0.2\%


Fan, Y., Xi, L., Hughes, D. S. T., Zhang, J., Zhang, J., Futreal, P. A., … Wang, W. (2016). MuSE: accounting for tumor heterogeneity using a sample-specific error model improves sensitivity and specificity in mutation calling from sequencing data. Genome Biology, 17(1), 178. \url{https://doi.org/10.1186/s13059-016-1029-6}

\begin{verbatim}
@article{MuSE,
author = {Fan, Yu and Xi, Liu and Hughes, Daniel S. T. and Zhang, Jianjun and
Zhang, Jianhua and Futreal, P. Andrew and Wheeler, David A. and Wang, Wenyi},
doi = {10.1186/s13059-016-1029-6},
issn = {1474-760X},
journal = {Genome Biology},
month = {dec},
number = {1},
pages = {178},
title = {{MuSE: accounting for tumor heterogeneity using a sample-specific error
model improves sensitivity and specificity in mutation calling from sequencing
data}},
url = {http://genomebiology.biomedcentral.com/articles/10.1186/s13059-016-1029-6},
volume = {17},
year = {2016}
}
\end{verbatim}

\section{GATK4}

Official: \url{https://software.broadinstitute.org/gatk/}

GitHub: \url{https://github.com/broadinstitute/gatk}

Conda: \url{https://anaconda.org/bioconda/gatk4}

License: BSD-3-Clause

Java 93.9\% Python 2.8\% wdl 1.8\% Shell 1.1\% R 0.3\% HTML 0.1\%


\textbf{The fourth paper, technically just a manuscript deposited in bioRxiv -- but it counts! This is a good citation to include in a Materials and Methods section or in a Discussion if you're talking about the joint calling process.}


Poplin, R., Ruano-Rubio, V., DePristo, M. A., Fennell, T. J., Carneiro, M. O., Auwera, G. A. Van der, … Banks, E. (2017). Scaling accurate genetic variant discovery to tens of thousands of samples. BioRxiv, 201178. \url{https://doi.org/10.1101/201178}

\begin{verbatim}
@article{GATK1,
abstract = {Comprehensive disease gene discovery in both common and rare
diseases will require the efficient and accurate detection of all classes of
genetic variation across tens to hundreds of thousands of human samples. We
describe here a novel assembly-based approach to variant calling, the GATK
HaplotypeCaller (HC) and Reference Confidence Model (RCM), that determines
genotype likelihoods independently per-sample but performs joint calling across
all samples within a project simultaneously. We show by calling over 90,000
samples from the Exome Aggregation Consortium (ExAC) that, in contrast to other
algorithms, the HC-RCM scales efficiently to very large sample sizes without
loss in accuracy; and that the accuracy of indel variant calling is superior
in comparison to other algorithms. More importantly, the HC-RCM produces a fully
squared-off matrix of genotypes across all samples at every genomic position
being investigated. The HCRCM is a novel, scalable, assembly-based algorithm
with abundant applications for population genetics and clinical studies.},
author = {Poplin, Ryan and Ruano-Rubio, Valentin and DePristo, Mark A. and
Fennell, Tim J. and Carneiro, Mauricio O. and der Auwera, Geraldine A. Van and
Kling, David E. and Gauthier, Laura D. and Levy-Moonshine, Ami and Roazen, David
and Shakir, Khalid and Thibault, Joel and Chandran, Sheila and Whelan, Chris and
Lek, Monkol and Gabriel, Stacey and Daly, Mark J. and Neale, Ben and MacArthur,
Daniel G. and Banks, Eric},
doi = {10.1101/201178},
journal = {bioRxiv},
pages = {201178},
title = {{Scaling accurate genetic variant discovery to tens of thousands of
samples}},
url = {https://www.biorxiv.org/content/10.1101/201178v2},
year = {2017}
}
\end{verbatim}

\textbf{The first GATK paper covers the computational philosophy underlying the GATK and is a good citation for the GATK in general.}


McKenna, A., Hanna, M., Banks, E., Sivachenko, A., Cibulskis, K., Kernytsky, A., … DePristo, M. A. (2010). The genome analysis toolkit: A MapReduce framework for analyzing next-generation DNA sequencing data. Genome Research, 20(9), 1297–1303. \url{https://doi.org/10.1101/gr.107524.110}

\begin{verbatim}
@article{GATK2,
abstract = {Next-generation DNA sequencing (NGS) projects, such as the 1000
Genomes Project, are already revolutionizing our understanding of genetic
variation among individuals. However, the massive data sets generated by NGS -
the 1000 Genome pilot alone includes nearly five terabases - make writing
feature-rich, efficient, and robust analysis tools difficult for even
computationally sophisticated individuals. Indeed, many professionals are
limited in the scope and the ease with which they can answer scientific
questions by the complexity of accessing and manipulating the data produced by
these machines. Here, we discuss our Genome Analysis Toolkit (GATK), a
structured programming framework designed to ease the development of efficient
and robust analysis tools for next-generation DNA sequencers using the
functional programming philosophy of MapReduce. The GATK provides a small but
rich set of data access patterns that encompass the majority of analysis tool
needs. Separating specific analysis calculations from common data management
infrastructure enables us to optimize the GATK framework for correctness,
stability, and CPU and memory efficiency and to enable distributed and shared
memory parallelization. We highlight the capabilities of the GATK by describing
the implementation and application of robust, scale-tolerant tools like coverage
calculators and single nucleotide polymorphism (SNP) calling. We conclude that
the GATK programming framework enables developers and analysts to quickly and
easily write efficient and robust NGS tools, many of which have already been
incorporated into large-scale sequencing projects like the 1000 Genomes Project
and The Cancer Genome Atlas. {\textcopyright} 2010 by Cold Spring Harbor
Laboratory Press.},
author = {McKenna, Aaron and Hanna, Matthew and Banks, Eric and Sivachenko,
Andrey and Cibulskis, Kristian and Kernytsky, Andrew and Garimella, Kiran and
Altshuler, David and Gabriel, Stacey and Daly, Mark and DePristo, Mark A.},
doi = {10.1101/gr.107524.110},
issn = {10889051},
journal = {Genome Research},
month = {sep},
number = {9},
pages = {1297--1303},
pmid = {20644199},
title = {{The genome analysis toolkit: A MapReduce framework for analyzing
next-generation DNA sequencing data}},
volume = {20},
year = {2010}
}
\end{verbatim}

\textbf{The second GATK paper describes in more detail some of the key tools commonly used in the GATK for high-throughput sequencing data processing and variant discovery. The paper covers base quality score recalibration, indel realignment, SNP calling with UnifiedGenotyper, variant quality score recalibration and their application to deep whole genome, whole exome, and low-pass multi-sample calling. This is a good citation if you use the GATK for variant discovery.}


Depristo, M. A., Banks, E., Poplin, R., Garimella, K. V., Maguire, J. R., Hartl, C., … Daly, M. J. (2011). A framework for variation discovery and genotyping using next-generation DNA sequencing data. Nature Genetics, 43(5), 491–501. \url{https://doi.org/10.1038/ng.806}

\begin{verbatim}
@article{GATK3,
abstract = {Recent advances in sequencing technology make it possible to
comprehensively catalog genetic variation in population samples, creating a
foundation for understanding human disease, ancestry and evolution. The amounts
of raw data produced are prodigious, and many computational steps are required
to translate this output into high-quality variant calls. We present a unified
analytic framework to discover and genotype variation among multiple samples
simultaneously that achieves sensitive and specific results across five sequencing
technologies and three distinct, canonical experimental designs. Our process
includes (i) initial read mapping; (ii) local realignment around indels; (iii)
base quality score recalibration; (iv) SNP discovery and genotyping to find
all potential variants; and (v) machine learning to separate true
segregating variation from machine artifacts common to next-generation
sequencing technologies. We here discuss the application of these tools,
instantiated in the Genome Analysis Toolkit, to deep whole-genome, whole-exome
capture and multi-sample low-pass (1/44X) 1000 Genomes Project datasets.
{\textcopyright} 2011 Nature America, Inc. All rights reserved.},
author = {Depristo, Mark A. and Banks, Eric and Poplin, Ryan and Garimella,
Kiran V. and Maguire, Jared R. and Hartl, Christopher and Philippakis, Anthony
A. and {Del Angel}, Guillermo and Rivas, Manuel A. and Hanna, Matt and McKenna,
Aaron and Fennell, Tim J. and Kernytsky, Andrew M. and Sivachenko, Andrey Y. and
Cibulskis, Kristian and Gabriel, Stacey B. and Altshuler, David and Daly, Mark J.},
doi = {10.1038/ng.806},
issn = {10614036},
journal = {Nature Genetics},
month = {may},
number = {5},
pages = {491--501},
pmid = {21478889},
title = {{A framework for variation discovery and genotyping using
next-generation DNA sequencing data}},
volume = {43},
year = {2011}
}
\end{verbatim}

\textbf{The third GATK paper describes the Best Practices for Variant Discovery (version 2.x). It is intended mainly as a learning resource for first-time users and as a protocol reference. This is a good citation to include in a Materials and Methods section.}

Van der Auwera, G. A., Carneiro, M. O., Hartl, C., Poplin, R., del Angel, G., Levy-Moonshine, A., … DePristo, M. A. (2013). From fastQ data to high-confidence variant calls: The genome analysis toolkit best practices pipeline. Current Protocols in Bioinformatics, (SUPL.43). \url{https://doi.org/10.1002/0471250953.bi1110s43}

\begin{verbatim}
@article{GATK4,
abstract = {This unit describes how to use BWA and the Genome Analysis Toolkit
(GATK) to map genome sequencing data to a reference and produce high-quality
variant calls that can be used in downstream analyses. The complete workflow
includes the core NGS dataprocessing steps that are necessary to make the
raw data suitable for analysis by the GATK, as well as the key methods
involved in variant discovery using the GATK. {\textcopyright} 2013 by John
Wiley {\&} Sons, Inc.},
author = {{Van der Auwera}, Geraldine A. and Carneiro, Mauricio O. and Hartl,
Christopher and Poplin, Ryan and del Angel, Guillermo and Levy-Moonshine, Ami and
Jordan, Tadeusz and Shakir, Khalid and Roazen, David and Thibault, Joel and
Banks, Eric and Garimella, Kiran V. and Altshuler, David and Gabriel, Stacey and
DePristo, Mark A.},
doi = {10.1002/0471250953.bi1110s43},
issn = {1934340X},
journal = {Current Protocols in Bioinformatics},
keywords = {Exome,Genotyping,NGS,Variant detection,WGS},
number = {SUPL.43},
pmid = {25431634},
publisher = {John Wiley and Sons Inc.},
title = {{From fastQ data to high-confidence variant calls: The genome analysis
toolkit best practices pipeline}},
year = {2013}
}
\end{verbatim}

\section{GATK3}

Conda: \url{https://anaconda.org/bioconda/gatk}

GoogleCloud: \url{https://console.cloud.google.com/storage/browser/gatk-software/package-archive/gatk/}

\section{MuTect1}

Official: \url{https://software.broadinstitute.org/cancer/cga/mutect}

GitHub: \url{https://github.com/broadinstitute/mutect}

Java 96.3\% Perl 2.3\% Scala 1.4\%

Need Java6/7. Cannot be run under Java 8.

Cibulskis, K., Lawrence, M. S., Carter, S. L., Sivachenko, A., Jaffe, D., Sougnez, C., … Getz, G. (2013). Sensitive detection of somatic point mutations in impure and heterogeneous cancer samples. Nature Biotechnology, 31(3), 213–219. \url{https://doi.org/10.1038/nbt.2514}

\begin{verbatim}
@article{MuTect,
abstract = {Detection of somatic point substitutions is a key step in
characterizing the cancer genome. However, existing methods typically miss
low-allelic-fraction mutations that occur in only a subset of the sequenced
cells owing to either tumor heterogeneity or contamination by normal cells. Here
we present MuTect, a method that applies a Bayesian classifier to detect somatic
mutations with very low allele fractions, requiring only a few supporting reads,
followed by carefully tuned filters that ensure high specificity. We also
describe benchmarking approaches that use real, rather than simulated, sequencing
data to evaluate the sensitivity and specificity as a function of sequencing
depth, base quality and allelic fraction. Compared with other methods, MuTect
has higher sensitivity with similar specificity, especially for mutations with
allelic fractions as low as 0.1 and below, making MuTect particularly useful for
studying cancer subclones and their evolution in standard exome and genome
sequencing data. Copyright {\textcopyright} 2013 Nature America, Inc.},
author = {Cibulskis, Kristian and Lawrence, Michael S. and Carter, Scott L. and
Sivachenko, Andrey and Jaffe, David and Sougnez, Carrie and Gabriel, Stacey and
Meyerson, Matthew and Lander, Eric S. and Getz, Gad},
doi = {10.1038/nbt.2514},
file = {::},
issn = {10870156},
journal = {Nature Biotechnology},
month = {mar},
number = {3},
pages = {213--219},
title = {{Sensitive detection of somatic point mutations in impure and
heterogeneous cancer samples}},
volume = {31},
year = {2013}
}
\end{verbatim}

\section{FreeBayes}

GitHub: \url{https://github.com/ekg/freebayes}

Conda: \url{https://anaconda.org/bioconda/freebayes}

License: MIT

C++ 95.7\% TeX 2.1\% Python 0.9\% Shell 0.5\% C 0.4\% Assembly 0.3\% Other 0.1\%

Garrison, E., \& Marth, G. (2012). Haplotype-based variant detection from short-read sequencing. Retrieved from \url{http://arxiv.org/abs/1207.3907}

\begin{verbatim}
@article{FreeBayes,
abstract = {The direct detection of haplotypes from short-read DNA sequencing
data requires changes to existing small-variant detection methods. Here, we
develop a Bayesian statistical framework which is capable of modeling
multiallelic loci in sets of individuals with non-uniform copy number. We then
describe our implementation of this framework in a haplotype-based variant
detector, FreeBayes.},
archivePrefix = {arXiv},
arxivId = {1207.3907},
author = {Garrison, Erik and Marth, Gabor},
eprint = {1207.3907},
month = {jul},
title = {{Haplotype-based variant detection from short-read sequencing}},
url = {http://arxiv.org/abs/1207.3907},
year = {2012}
}
\end{verbatim}

\section{Manta}

Conda: \url{https://anaconda.org/bioconda/manta}

GitHub: \url{https://github.com/Illumina/manta}

License: GPL 3

C++ 87.3\% Python 7.2\% CMake 4.3\% Shell 0.8\% Dockerfile 0.3\% C 0.1\%

Chen, X., Schulz-Trieglaff, O., Shaw, R., Barnes, B., Schlesinger, F., Källberg, M., … Saunders, C. T. (2016). Manta: Rapid detection of structural variants and indels for germline and cancer sequencing applications. Bioinformatics, 32(8), 1220–1222. \url{https://doi.org/10.1093/bioinformatics/btv710}

\begin{verbatim}
@article{manta,
abstract = {We describe Manta, a method to discover structural variants and
indels from next generation sequencing data. Manta is optimized for rapid
germline and somatic analysis, calling structural variants, medium-sized indels
and large insertions on standard compute hardware in less than a tenth of the
time that comparable methods require to identify only subsets of these variant
types: for example NA12878 at 50× genomic coverage is analyzed in less than 20
min. Manta can discover and score variants based on supporting paired and
split-read evidence, with scoring models optimized for germline analysis of
diploid individuals and somatic analysis of tumor-normal sample pairs. Call
quality is similar to or better than comparable methods, as determined by
pedigree consistency of germline calls and comparison of somatic calls to COSMIC
database variants. Manta consistently assembles a higher fraction of its calls
to base-pair resolution, allowing for improved downstream annotation and
analysis of clinical significance. We provide Manta as a community resource to f
acilitate practical and routine structural variant analysis in clinical and
research sequencing scenarios.},
author = {Chen, Xiaoyu and Schulz-Trieglaff, Ole and Shaw, Richard and Barnes,
Bret and Schlesinger, Felix and K{\"{a}}llberg, Morten and Cox, Anthony J. and
Kruglyak, Semyon and Saunders, Christopher T.},
doi = {10.1093/bioinformatics/btv710},
issn = {14602059},
journal = {Bioinformatics},
month = {apr},
number = {8},
pages = {1220--1222},
pmid = {26647377},
publisher = {Oxford University Press},
title = {{Manta: Rapid detection of structural variants and indels for germline
and cancer sequencing applications}},
volume = {32},
year = {2016}
}
\end{verbatim}

\section{Shimmer}

GitHub: \url{https://github.com/nhansen/Shimmer}

License: Public Domain -- United States Government Work

Perl 68.1\% C 11.4\% Python 11.0\% R 7.9\% Other 1.0\% Makefile 0.5\% Shell 0.1\%

Hansen, N. F., Gartner, J. J., Mei, L., Samuels, Y., \& Mullikin, J. C. (2013). Shimmer: Detection of genetic alterations in tumors using next-generation sequence data. Bioinformatics, 29(12), 1498–1503. \url{https://doi.org/10.1093/bioinformatics/btt183}

\begin{verbatim}
@article{shimmer,
abstract = {Motivation: Extensive DNA sequencing of tumor and matched normal
samples using exome and whole-genome sequencing technologies has enabled the
discovery of recurrent genetic alterations in cancer cells, but variability in
stromal contamination and subclonal heterogeneity still present a severe
challenge to available detection algorithms.Results: Here, we describe publicly
available software, Shimmer, which accurately detects somatic single-nucleotide
variants using statistical hypothesis testing with multiple testing correction.
This program produces somatic single-nucleotide variant predictions with
significantly higher sensitivity and accuracy than other available software when
run on highly contaminated or heterogeneous samples, and it gives comparable
sensitivity and accuracy when run on samples of high purity. {\textcopyright}
2012 The Author.},
author = {Hansen, Nancy F. and Gartner, Jared J. and Mei, Lan and Samuels,
Yardena and Mullikin, James C.},
doi = {10.1093/bioinformatics/btt183},
issn = {13674803},
journal = {Bioinformatics},
month = {jun},
number = {12},
pages = {1498--1503},
title = {{Shimmer: Detection of genetic alterations in tumors using
next-generation sequence data}},
volume = {29},
year = {2013}
}
\end{verbatim}

\section{Strelka2}

Conda: \url{https://anaconda.org/bioconda/strelka}

GitHub: \url{https://github.com/Illumina/strelka}

License: GPL 3

C++ 81.3\% Python 12.8\% CMake 4.2\% Shell 1.1\% Dockerfile 0.3\% Jupyter Notebook 0.2\% C 0.1\%

Kim, S., Scheffler, K., Halpern, A. L., Bekritsky, M. A., Noh, E., Källberg, M., … Saunders, C. T. (2018). Strelka2: fast and accurate calling of germline and somatic variants. Nature Methods, 15(8), 591–594. \url{https://doi.org/10.1038/s41592-018-0051-x}

\begin{verbatim}
@article{Strelka2,
abstract = {We describe Strelka2 (https://github.com/Illumina/strelka), an
open-source small-variant-calling method for research and clinical germline and
somatic sequencing applications. Strelka2 introduces a novel mixture-model-based
estimation of insertion/deletion error parameters from each sample, an efficient
tiered haplotype-modeling strategy, and a normal sample contamination model to
improve liquid tumor analysis. For both germline and somatic calling, Strelka2
substantially outperformed the current leading tools in terms of both
variant-calling accuracy and computing cost.},
author = {Kim, Sangtae and Scheffler, Konrad and Halpern, Aaron L. and
Bekritsky, Mitchell A. and Noh, Eunho and K{\"{a}}llberg, Morten and Chen,
Xiaoyu and Kim, Yeonbin and Beyter, Doruk and Krusche, Peter and Saunders,
Christopher T.},
doi = {10.1038/s41592-018-0051-x},
issn = {15487105},
journal = {Nature Methods},
month = {aug},
number = {8},
pages = {591--594},
publisher = {Nature Publishing Group},
title = {{Strelka2: fast and accurate calling of germline and somatic variants}},
volume = {15},
year = {2018}
}

\end{verbatim}



\part{VCF/BCF Processor}

\section{BCFTools}

Official: \url{http://www.htslib.org/}

Official: \url{http://samtools.github.io/bcftools/}

GitHub: \url{https://github.com/samtools/bcftools}

Conda: \url{https://anaconda.org/bioconda/bcftools}

License: MIT/Expat license or GPL 3

C 87.6\% Perl 9.5\% M4 0.9\% Python 0.7\% Makefile 0.6\% Shell 0.6\% Other 0.1\%

Same as Samtools.

\section{VCFTools}

SourceForge: \url{http://vcftools.sourceforge.net/}

Official: \url{http://vcftools.github.io/}

GitHub: \url{https://github.com/vcftools/vcftools}

Conda: \url{https://anaconda.org/bioconda/vcftools}

License: LGPL 3

C++ 44.8\% Perl 44.5\% C 6.6\% Roff 3.0\% Shell 0.8\% M4 0.2\% Makefile 0.1\%



Danecek, P., Auton, A., Abecasis, G., Albers, C. A., Banks, E., DePristo, M. A., … Durbin, R. (2011). The variant call format and VCFtools. Bioinformatics, 27(15), 2156–2158. \url{https://doi.org/10.1093/bioinformatics/btr330}

\begin{verbatim}
@article{VCFTools,
abstract = {Summary: The variant call format (VCF) is a generic format for
storing DNA polymorphism data such as SNPs, insertions, deletions and structural
variants, together with rich annotations. VCF is usually stored in a compressed
manner and can be indexed for fast data retrieval of variants from a range of
positions on the reference genome. The format was developed for the 1000 Genomes
Project, and has also been adopted by other projects such as UK10K, dbSNP and
the NHLBI Exome Project. VCFtools is a software suite that implements various
utilities for processing VCF files, including validation, merging, comparing and
also provides a general Perl API. {\textcopyright} The Author(s) 2011. Published
by Oxford University Press.},
author = {Danecek, Petr and Auton, Adam and Abecasis, Goncalo and Albers,
Cornelis A. and Banks, Eric and DePristo, Mark A. and Handsaker, Robert E. and
Lunter, Gerton and Marth, Gabor T. and Sherry, Stephen T. and McVean, Gilean and
Durbin, Richard},
doi = {10.1093/bioinformatics/btr330},
issn = {13674803},
journal = {Bioinformatics},
month = {aug},
number = {15},
pages = {2156--2158},
pmid = {21653522},
title = {{The variant call format and VCFtools}},
volume = {27},
year = {2011}
}
\end{verbatim}

\part{VCF Annotation \& Visualization}

\section{VarSifter}

GitHub: \url{https://github.com/teerjk/VarSifter}

Old Official: \url{research.nhgri.nih.gov/software/VarSifter/}

License: THE JUNG LICENSE (Another kind of BSD)

Java 91.1\% HTML 8.4\% Makefile 0.5\%



Teer, J. K., Green, E. D., Mullikin, J. C., \& Biesecker, L. G. (2012). VarSifter: Visualizing and analyzing exome-scale sequence variation data on a desktop computer. Bioinformatics, 28(4), 599–600. \url{https://doi.org/10.1093/bioinformatics/btr711}

\begin{verbatim}
@article{VarSifter,
abstract = {VarSifter is a graphical software tool for desktop computers that
allows investigators of varying computational skills to easily and quickly sort,
filter, and sift through sequence variation data. A variety of filters and a
custom query framework allow filtering based on any combination of sample and
annotation information. By simplifying visualization and analyses of exome-scale
sequence variation data, this program will help bring the power and promise of
massively-parallel DNA sequencing to a broader group of researchers. Published
by Oxford University Press 2012.},
author = {Teer, Jamie K. and Green, Eric D. and Mullikin, James C. and
Biesecker, Leslie G.},
doi = {10.1093/bioinformatics/btr711},
issn = {13674803},
journal = {Bioinformatics},
month = {feb},
number = {4},
pages = {599--600},
title = {{VarSifter: Visualizing and analyzing exome-scale sequence variation
data on a desktop computer}},
volume = {28},
year = {2012}
}
\end{verbatim}

\section{ANNOVar}

Official: \url{http://www.openbioinformatics.org/annovar/annovar_download.html}

Wang, K., Li, M., \& Hakonarson, H. (2010). ANNOVAR: functional annotation of genetic variants from high-throughput sequencing data. Nucleic Acids Research, 38(16), e164–e164. \url{https://doi.org/10.1093/nar/gkq603}

\begin{verbatim}
@article{ANNOVar,
author = {Wang, K. and Li, M. and Hakonarson, H.},
doi = {10.1093/nar/gkq603},
issn = {0305-1048},
journal = {Nucleic Acids Research},
month = {sep},
number = {16},
pages = {e164--e164},
title = {{ANNOVAR: functional annotation of genetic variants from high-throughput
sequencing data}},
url = {https://academic.oup.com/nar/article-lookup/doi/10.1093/nar/gkq603},
volume = {38},
year = {2010}
}

\end{verbatim}

\section{Variant Effect Predicator}

GitHub \url{https://github.com/Ensembl/ensembl-vep}

Official: \url{https://asia.ensembl.org/info/docs/tools/vep/index.html}

Conda: \url{https://anaconda.org/bioconda/ensembl-vep}

License: Apache 2.0

Perl 98.4\% Other 1.6\%



McLaren, W., Gil, L., Hunt, S. E., Riat, H. S., Ritchie, G. R. S., Thormann, A., Flicek, P. and Cunningham, F. (2016) The Ensembl Variant Effect Predictor, Genome Biology. BioMed Central Ltd., 17(1), p. 122. doi: 10.1186/s13059-016-0974-4.

\begin{verbatim}
@article{VEP,
abstract = {The Ensembl Variant Effect Predictor is a powerful toolset for the
analysis, annotation, and prioritization of genomic variants in coding and
non-coding regions. It provides access to an extensive collection of genomic
annotation, with a variety of interfaces to suit different requirements, and
simple options for configuring and extending analysis. It is open source, free
to use, and supports full reproducibility of results. The Ensembl Variant Effect
Predictor can simplify and accelerate variant interpretation in a wide range of
study designs.},
author = {McLaren, William and Gil, Laurent and Hunt, Sarah E. and Riat, Harpreet
Singh and Ritchie, Graham R.S. and Thormann, Anja and Flicek, Paul and
Cunningham, Fiona},
doi = {10.1186/s13059-016-0974-4},
issn = {1474760X},
journal = {Genome Biology},
keywords = {Genome,NGS,SNP,Variant annotation},
month = {jun},
number = {1},
pages = {122},
pmid = {27268795},
publisher = {BioMed Central Ltd.},
title = {{The Ensembl Variant Effect Predictor}},
url = {http://genomebiology.biomedcentral.com/articles/10.1186/s13059-016-0974-4},
volume = {17},
year = {2016}
}
\end{verbatim}

\section{snpEff}

Official: \url{http://snpeff.sourceforge.net/}

GitHub: \url{https://github.com/pcingola/SnpEff}

Conda: \url{https://anaconda.org/bioconda/snpeff}

License: LGPL 3

Java 62.0\% HTML 31.5\% Perl 2.2\% Shell 1.7\% CSS 0.8\% Python 0.6\% Other 1.2\%



Cingolani, P., Platts, A., Wang, L. L., Coon, M., Nguyen, T., Wang, L., Land, S. J., Lu, X. and Ruden, D. M. (2012) A program for annotating and predicting the effects of single nucleotide polymorphisms, SnpEff: SNPs in the genome of Drosophila melanogaster strain w1118; iso-2; iso-3, Fly. Taylor and Francis Inc., 6(2), pp. 80–92. doi: 10.4161/fly.19695.

\begin{verbatim}
@article{snpEff,
abstract = {We describe a new computer program, SnpEff, for rapidly categorizing
the effects of variants in genome sequences. Once a genome is sequenced, SnpEff
annotates variants based on their genomic locations and predicts coding effects.
Annotated genomic locations include intronic, untranslated region, upstream,
downstream, splice site, or intergenic regions. Coding effects such as synonymous
or non-synonymous amino acid replacement, start codon gains or losses, stop
codon gains or losses, or frame shifts can be predicted. Here the use of SnpEff
is illustrated by annotating {\~{}}356,660 candidate SNPs in {\~{}}117 Mb unique
sequences, representing a substitution rate of {\~{}}1/305 nucleotides, between
the Drosophila melanogaster w1118; iso-2; iso-3 strain and the reference y1; cn1
bw1 sp1 strain. We show that {\~{}}15,842 SNPs are synonymous and {\~{}}4,467
SNPs are non-synonymous (N/S {\~{}}0.28). The remaining SNPs are in other
categories, such as stop codon gains (38 SNPs), stop codon losses (8 SNPs), and
start codon gains (297 SNPs) in the 5'UTR. We found, as expected, that the SNP
frequency is proportional to the recombination frequency (i.e., highest in the
middle of chromosome arms). We also found that start-gain or stop-lost SNPs in
Drosophila melanogaster often result in additions of N-terminal or C-terminal
amino acids that are conserved in other Drosophila species. It appears that the
5' and 3'UTRs are reservoirs for genetic variations that changes the termini of
proteins during evolution of the Drosophila genus. As genome sequencing is
becoming inexpensive and routine, SnpEff enables rapid analyses of whole-genome
sequencing data to be performed by an individual laboratory. {\textcopyright}
2012 Landes Bioscience.},
author = {Cingolani, Pablo and Platts, Adrian and Wang, Le Lily and Coon,
Melissa and Nguyen, Tung and Wang, Luan and Land, Susan J. and Lu, Xiangyi and
Ruden, Douglas M.},
doi = {10.4161/fly.19695},
issn = {19336942},
journal = {Fly},
keywords = {Drosophila melanogaster,Next generation DNA sequencing,Personal
genomes,Whole-genome SNP analysis},
month = {apr},
number = {2},
pages = {80--92},
pmid = {22728672},
publisher = {Taylor and Francis Inc.},
title = {{A program for annotating and predicting the effects of single
nucleotide polymorphisms, SnpEff: SNPs in the genome of Drosophila melanogaster
strain w1118; iso-2; iso-3}},
url = {http://www.tandfonline.com/doi/abs/10.4161/fly.19695},
volume = {6},
year = {2012}
}
\end{verbatim}

\section{snpSift}

Official: \url{http://snpeff.sourceforge.net/}

GitHub: \url{https://github.com/pcingola/SnpSift}

Conda: \url{https://anaconda.org/bioconda/snpsift}

License: LGPL 3

Java 97.3\% G-code 2.3\% Shell 0.4\%



Cingolani, P., Patel, V. M., Coon, M., Nguyen, T., Land, S. J., Ruden, D. M. and Lu, X. (2012) Using Drosophila melanogaster as a model for genotoxic chemical mutational studies with a new program, SnpSift, Frontiers in Genetics, 3(MAR). doi: 10.3389/fgene.2012.00035.

\begin{verbatim}

@article{snpSift,
abstract = {This paper describes a new program SnpSift for filtering
differential DNA sequence variants between two or more experimental genomes
after genotoxic chemical exposure. Here, we illustrate how SnpSift can be used
to identify candidate phenotype-relevant variants including single nucleotide
polymorphisms, multiple nucleotide polymorphisms, insertions, and deletions
(InDels) in mutant strains isolated from genome-wide chemical mutagenesis of
Drosophila melanogaster. First, the genomes of two independently isolated mutant
fly strains that are allelic for a novel recessive male-sterile locus generated
by genotoxic chemical exposure were sequenced using the Illumina next-generation
DNA sequencer to obtain 20- to 29-fold coverage of the euchromatic sequences.
The sequencing reads were processed and variants were called using standard
bioinformatic tools. Next, SnpEff was used to annotate all sequence variants and
their potential mutational effects on associated genes. Then, SnpSift was used
to filter and select differential variants that potentially disrupt a common
gene in the two allelic mutant strains. The potential causative DNA lesions were
partially validated by capillary sequencing of polymerase chain
reaction-amplified DNA in the genetic interval as defined by meiotic mapping and
deletions that remove defined regions of the chromosome. Of the five candidate
genes located in the genetic interval, the Pka-like gene CG12069 was found to
carry a separate pre-mature stop codon mutation in each of the two allelic
mutants whereas the other four candidate genes within the interval have wild-type
sequences. The Pka-like gene is therefore a strong candidate gene for the
male-sterile locus. These results demonstrate that combining SnpEff and SnpSift
can expedite the identification of candidate phenotype-causative mutations in
chemically mutagenized Drosophila strains. This technique can also be used to
characterize the variety of mutations generated by genotoxic chemicals.
{\textcopyright} 2012 Cingolani, Patel, Coon, Nguyen, Land, Ruden and Lu.},
author = {Cingolani, Pablo and Patel, Viral M. and Coon, Melissa and Nguyen, Tung
and Land, Susan J. and Ruden, Douglas M. and Lu, Xiangyi},
doi = {10.3389/fgene.2012.00035},
issn = {16648021},
journal = {Frontiers in Genetics},
keywords = {Drosophila melanogaster,Next-generation DNA sequencing,Personal
genomes,Whole-genome SNP analysis},
number = {MAR},
title = {{Using Drosophila melanogaster as a model for genotoxic chemical
mutational studies with a new program, SnpSift}},
volume = {3},
year = {2012}
}
\end{verbatim}

\section{myVCF}

Official: \url{https://apietrelli.github.io/myVCF/}

ReadTheDocs: \url{https://myvcf.readthedocs.io/en/latest/}

GitHub: \url{https://github.com/apietrelli/myVCF}

License: CC-BY-NC 4.0 International

Python 93.7\% HTML 6.2\% Other 0.1\%

Pietrelli, A. and Valenti, L. (2017) myVCF: a desktop application for high-throughput mutations data management, Bioinformatics, 33(22), pp. 3676–3678. doi: 10.1093/bioinformatics/btx475.

\begin{verbatim}
@article{myVCF,
abstract = {Next-generation sequencing technologies have become the most
powerful tool to discover genetic variants associated with human diseases.
Although the dramatic reductions in the costs facilitate the use in the wet-lab
and clinics, the huge amount of data generated renders their management by
non-expert researchers and physicians extremely difficult. Therefore, there is
an urgent need of novel approaches and tools aimed at getting the ‘end-users'
closer to the sequencing data, facilitating the access by non-bioinformaticians,
and to speed-up the functional interpretation of genetic variants. We developed
myVCF, a standalone, easy-to-use desktop application, which is based on a browser
interface and is suitable for Windows, Mac and UNIX systems. myVCF is an
efficient platform that is able to manage multiple sequencing projects created
from VCF files within the system; stores genetic variants and samples genotypes
from an annotated VCF files into a SQLite database; implements a flexible search
engine for data exploration, allowing to query for chromosomal region, gene,
single variant or dbSNP ID. Besides, myVCF generates a summary statistics report
about mutations distribution across samples and across the genome/exome by
aggregating the information within the VCF file. In summary, the myVCF platform
allows end-users without strong programming and bioinformatics skills to
explore, query, visualize and export mutations data in a simple and
straightforward way.https://apietrelli.github.io/myVCF/Supplementary data
are available at Bioinformatics online.},
author = {Pietrelli, Alessandro and Valenti, Luca},
doi = {10.1093/bioinformatics/btx475},
issn = {1367-4803},
journal = {Bioinformatics},
month = {jul},
number = {22},
pages = {3676--3678},
title = {{myVCF: a desktop application for high-throughput mutations data
management}},
url = {https://doi.org/10.1093/bioinformatics/btx475},
volume = {33},
year = {2017}
}


\end{verbatim}



\part{Pipelines}
\section{SpeedSeq}

GitHub: \url{https://github.com/hall-lab/speedseq}

License: The MIT License (MIT)

C 75.5\% Roff 8.4\% Perl 8.0\% Shell 3.1\% Makefile 1.5\% C++ 1.0\% Other 2.5\%



Chiang, C., Layer, R. M., Faust, G. G., Lindberg, M. R., Rose, D. B., Garrison, E. P., Marth, G. T., Quinlan, A. R. and Hall, I. M. (2015) SpeedSeq: Ultra-fast personal genome analysis and interpretation, Nature Methods. Nature Publishing Group, 12(10), pp. 966–968. doi: 10.1038/nmeth.3505.

\begin{verbatim}
@article{Chiang2015,
abstract = {SpeedSeq is an open-source genome analysis platform that
accomplishes alignment, variant detection and functional annotation of a 50×
human genome in 13 h on a low-cost server and alleviates a bioinformatics
bottleneck that typically demands weeks of computation with extensive hands-on
expert involvement. SpeedSeq offers performance competitive with or superior to
current methods for detecting germline and somatic single-nucleotide variants,
structural variants, insertions and deletions, and it includes novel
functionality for streamlined interpretation.},
author = {Chiang, Colby and Layer, Ryan M. and Faust, Gregory G. and Lindberg,
Michael R. and Rose, David B. and Garrison, Erik P. and Marth, Gabor T. and
Quinlan, Aaron R. and Hall, Ira M.},
doi = {10.1038/nmeth.3505},
issn = {15487105},
journal = {Nature Methods},
month = {sep},
number = {10},
pages = {966--968},
publisher = {Nature Publishing Group},
title = {{SpeedSeq: Ultra-fast personal genome analysis and interpretation}},
volume = {12},
year = {2015}
}
\end{verbatim}

\part{R, IDE and R Packages}

\section{R}

Conda: \url{https://anaconda.org/r/r-base}

Official: \url{https://www.r-project.org/}

SVN: \url{https://svn.r-project.org/R-dev-web/trunk/index.html}

License: GPL 2

R Core Team (2020) R: A Language and Environment for Statistical Computing. Vienna, Austria. Available at: \url{https://www.r-project.org}.

\begin{verbatim}
@Manual{R,
title        = {R: A Language and Environment for Statistical
Computing},
author       = {{R Core Team}},
organization = {R Foundation for Statistical Computing},
address      = {Vienna, Austria},
year         = {2020},
url          = {https://www.R-project.org}
}
\end{verbatim}

\section{RStudio}

Official: \url{https://rstudio.com/}

Conda: \url{https://anaconda.org/r/rstudio}

GitHub: \url{https://github.com/rstudio/rstudio}

License: Commercial or AGPL 3

Java 36.4\% C++ 32.2\% JavaScript 19.5\% R 3.4\% C 3.3\% TypeScript 1.9\% Other 3.3\% 

RStudio Team (2015). RStudio: Integrated Development for R. RStudio, Inc., Boston, MA URL http://www.rstudio.com/.

\begin{verbatim}
@Manual{,
title = {RStudio: Integrated Development Environment for R},
author = {{RStudio Team}},
organization = {RStudio, Inc.},
address = {Boston, MA},
year = {2015},
url = {http://www.rstudio.com/},
}
\end{verbatim}




\part{Python and Python Packages}

\section{Python3}

Conda: \url{https://anaconda.org/conda-forge/python}

Official: \url{https://www.python.org/}

GitHub: \url{https://github.com/python/cpython}

License: PYTHON SOFTWARE FOUNDATION LICENSE VERSION 2

Python 64.3\% C 28.7\% Objective-C 4.3\% C++ 1.2\% HTML 0.4\% M4 0.4\% Other 0.7\% 

\textbf{From General Python 3 FAQ: \url{https://docs.python.org/3/faq/general.html}}

Are there any published articles about Python that I can reference?

It's probably best to cite your favourite book about Python.

The very first article about Python was written in 1991 and is now quite outdated.

Rossum, G. van and Boer, J. de (1991) Interactively Testing Remote Servers Using the Python Programming Language, CWI Quarterly, pp. 283–303.

\begin{verbatim}
@misc{Rossum1991,
abstract = {This paper describes how two tools that were developed quite 
independently gained in power by a well-designed connection between them. The 
tools are Python, an interpreted prototyping language and AIL, a Remote
Procedure Call stub generator. The context is Amoeba, a well-known distributed
operating system developed jointly by the Free University and CWI in Amsterdam.
As a consequence of their integration, both tools have profited: Python gained
usability whan used with Amoeba - for which it was not specifically developed
- and AIL users now have a powerful interactive tool to test servers and to
experiment with new client/server interfaces.},
author = {van Rossum, Guido and de Boer, Jelke},
booktitle = {CWI Quarterly},
issn = {0922-5366},
keywords = {python},
pages = {283--303},
title = {{Interactively Testing Remote Servers Using the Python Programming
Language}},
year = {1991}
}
\end{verbatim}

\part{Docker and Docker Packages}




\end{document}
